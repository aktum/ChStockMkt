%Header
\documentclass[a4paper,12pt]{article}

\usepackage[T1,T2A]{fontenc} % make cyrillic symbols work
\usepackage[utf8]{inputenc}
\usepackage[english,russian]{babel}
\usepackage{cmap}					% enable search in PDF

% Packages to include R output in the file
\usepackage{color}
\usepackage{fancyvrb}
\newcommand{\VerbBar}{|}
\newcommand{\VERB}{\Verb[commandchars=\\\{\}]}
\DefineVerbatimEnvironment{Highlighting}{Verbatim}{commandchars=\\\{\},fontsize=\small}
% Add ',fontsize=\small' for more characters per line
\usepackage{framed}
\definecolor{shadecolor}{RGB}{248,248,248}
\newenvironment{Shaded}{\begin{snugshade}}{\end{snugshade}}
\newcommand{\AlertTok}[1]{\textcolor[rgb]{0.94,0.16,0.16}{#1}}
\newcommand{\AnnotationTok}[1]{\textcolor[rgb]{0.56,0.35,0.01}{\textbf{\textit{#1}}}}
\newcommand{\AttributeTok}[1]{\textcolor[rgb]{0.77,0.63,0.00}{#1}}
\newcommand{\BaseNTok}[1]{\textcolor[rgb]{0.00,0.00,0.81}{#1}}
\newcommand{\BuiltInTok}[1]{#1}
\newcommand{\CharTok}[1]{\textcolor[rgb]{0.31,0.60,0.02}{#1}}
\newcommand{\CommentTok}[1]{\textcolor[rgb]{0.56,0.35,0.01}{\textit{#1}}}
\newcommand{\CommentVarTok}[1]{\textcolor[rgb]{0.56,0.35,0.01}{\textbf{\textit{#1}}}}
\newcommand{\ConstantTok}[1]{\textcolor[rgb]{0.00,0.00,0.00}{#1}}
\newcommand{\ControlFlowTok}[1]{\textcolor[rgb]{0.13,0.29,0.53}{\textbf{#1}}}
\newcommand{\DataTypeTok}[1]{\textcolor[rgb]{0.13,0.29,0.53}{#1}}
\newcommand{\DecValTok}[1]{\textcolor[rgb]{0.00,0.00,0.81}{#1}}
\newcommand{\DocumentationTok}[1]{\textcolor[rgb]{0.56,0.35,0.01}{\textbf{\textit{#1}}}}
\newcommand{\ErrorTok}[1]{\textcolor[rgb]{0.64,0.00,0.00}{\textbf{#1}}}
\newcommand{\ExtensionTok}[1]{#1}
\newcommand{\FloatTok}[1]{\textcolor[rgb]{0.00,0.00,0.81}{#1}}
\newcommand{\FunctionTok}[1]{\textcolor[rgb]{0.00,0.00,0.00}{#1}}
\newcommand{\ImportTok}[1]{#1}
\newcommand{\InformationTok}[1]{\textcolor[rgb]{0.56,0.35,0.01}{\textbf{\textit{#1}}}}
\newcommand{\KeywordTok}[1]{\textcolor[rgb]{0.13,0.29,0.53}{\textbf{#1}}}
\newcommand{\NormalTok}[1]{#1}
\newcommand{\OperatorTok}[1]{\textcolor[rgb]{0.81,0.36,0.00}{\textbf{#1}}}
\newcommand{\OtherTok}[1]{\textcolor[rgb]{0.56,0.35,0.01}{#1}}
\newcommand{\PreprocessorTok}[1]{\textcolor[rgb]{0.56,0.35,0.01}{\textit{#1}}}
\newcommand{\RegionMarkerTok}[1]{#1}
\newcommand{\SpecialCharTok}[1]{\textcolor[rgb]{0.00,0.00,0.00}{#1}}
\newcommand{\SpecialStringTok}[1]{\textcolor[rgb]{0.31,0.60,0.02}{#1}}
\newcommand{\StringTok}[1]{\textcolor[rgb]{0.31,0.60,0.02}{#1}}
\newcommand{\VariableTok}[1]{\textcolor[rgb]{0.00,0.00,0.00}{#1}}
\newcommand{\VerbatimStringTok}[1]{\textcolor[rgb]{0.31,0.60,0.02}{#1}}
\newcommand{\WarningTok}[1]{\textcolor[rgb]{0.56,0.35,0.01}{\textbf{\textit{#1}}}}

\usepackage{multicol} % multiple columns in part of the file

\usepackage{amsmath,amsfonts,amssymb,amsthm,mathtools} % AMS

\usepackage{chngcntr}  % restart equation numbering in each section, subsection
\counterwithin*{equation}{section}
\counterwithin*{equation}{subsection}

\usepackage{geometry} % Setting margins

\usepackage{booktabs}

\geometry{top=20mm}

\geometry{bottom=20mm}

\geometry{left=20mm}

\geometry{right=20mm}

\newtheorem{definition}{Определение}

\setlength{\parindent}{1.27 cm} % start each paragraph with an indent

\usepackage{setspace} % spacing options
\singlespacing


\title{Являются ли рынки акций КНР эффективными?}

\author{Туманянц Артемий}

\date{2019\\ВШЭ\\Москва}

\begin{document}
\maketitle

%Объем работы: минимум 15 страниц.
%Шрифт 12, интервал 1, выравнивание: по ширине
%Предполагается, что эта работа - Ваше авторское исследование выбранной проблематики:
%- введением на 1-2 стр. (актуальность, степень научной проработанности, объект, предмет, цель, задачи, методология расчетов, практическая значимость),
%- основной частью из нескольких глав (анализ и систематизация имеющихся научных работ (теоретическая база), описание выборки и методологии, расчетная часть),
%- заключением на 1-2 стр. (анализ и интерпретация полученных результатов, направления дальнейших исследований),
%- списком литературы,
%- приложением (если требуется).

\newpage
\tableofcontents
\newpage
\section{Введение}

Последние полвека любой курс или учебник по финансам обязательно включает в себя раздел, посвященный гипотезе о рыночной эффективности. Уже более 70 лет ученые используют различные методы для проверки рынков на эффективность. Однако большинство дискурса на эту тему ведется между западными экономистами и их работы в основном основаны на данных американских и европейских площадок. Ключевыми голосами в этом дискурсе являются лауреат Нобелевской премии по экономике 2013~г., Юджин Фама, автор популярной книги "Случайное блуждание по Уолл Стрит", Бёртон Малкил, и автор нескольких учебников по корпоративным финансам, Майк Дженсен. Джозеф Стиглитц, другой лауреат Нобелевской премии также писал о гипотезе об эффективности рынков. Назадолго до вручения Нобелевской премии Ричарду Таллеру университет Чикаго провел дебаты между Ю.~Фамой и Р.~Таллером. Последний является сторонником поведенческой теории в финансах, которая противоречит предпосылкам гипотезы об эффективности рынков. Запись дебатов посмотрело около 70 тыс. человек. Таким образом, на сегодняшний момент вопрос об эффективности рынков остается актуальным, несмотря на большую проработанность темы.

Данное исследование стремится использовать опыт исследования данной темы в западных развитых странах для определения эффективности китайского фондового рынка. Объектом явлются две биржевые площадки материкового Китая, Шанхайская и Шэньчжэньская фондовые биржи. Предмет исследования заключается в соответствии китайского рынка акций гипотезе об эффективности финансового рынка.

Цель состоит в проведении ряда тестов на соответствие динамики индексов Шэньчжэньской и Шанхайской бирж гипотезе о случайном блуждании котировок и о стационарности временного ряда. Исходя из поставленной цели можно поставить несколько задач:
\begin{itemize}
  \item Проанализировать теоретическую основу исследований об эффективности рынка;
  \item Провести тест на соответствие китайского фондового рынка гипотезе о случайном блужданнии;
  \item Провести ряд тестов на стационарность временного ряда цен закрытия фондовых бирж Китая;
  \item Проанализировать полученные результаты.
\end{itemize}

В основу методологии легла статистика отношения правдоподобия для авторегрессионных рядов с единичным корнем, описанная в \cite{Dickey1981}. Также использовались три теста на стационарность временного ряда: расширенный тест Дики-Фуллера, тест Филлипса-Перрона и KPSS тест. Исходный код, использованный в исследовании размещен в свободном доступе{\footnote{URL: https://github.com/aktum/ChStockMkt}}

Исходя из поставленных задач была определена следующая структура работы: глава \ref{sec:litrev} представляет теоретическую базу исследования, глава \ref{sec:meth} содержит описание данных и методологии, в главе \ref{sec:quant} перечислены результаты расчетов, а заключительная глава дает их интерпретацию и выводы.
\newpage
\section{Теоретическая база}\label{sec:litrev}

В данной работе проводится тест на выполнение гипотезы о "случайном блуждании", которая была разработана в 1950-1960-е годы и легла в основу гипотезы об эффективности рынка. \cite{Jensen1978} дает следующее определение эффективности рынка:

\begin{definition}
  Рынок считается эффективным по отношению к информационному множеству $\Omega$, если стратегия торговли, основанная на $\Omega$ не позволяет получать экономическую прибыль, т.е. положительную чистую выручку, скорректированную на риск.
\end{definition}

Выделяют три формы эффективности рынка в зависимости от содержания информационного множества $\Omega$:
\begin{enumerate}
  \item Слабая форма эффективности: в $\Omega$ заложены исторические данные об уровнях цен;
  \item Средняя форма эффективности: в $\Omega$ содержится вся публично доступная информация;
  \item Сильная форма эффективности: в $\Omega$ включены все данные известные кому-либо.
\end{enumerate}

Для целей данной работы наибольший интерес представляет слабая форма эффективности. \cite{Kendall1953} исследовал корреляцию между первыми разностями стоимости акций 19 компаний~(486 наблюдений) и 3 сырьевых индексов~(2387, 548 и 1446 наблюдений). Один из выводов анализа гласит, что данные ведут себя почти как блуждающие временные ряды. \cite{Granger1963} использовали спектральный анализ для проверки гипотезы о "случайном блуждании" на различных индексах финансовых рынков США, включая Dow-Jones Industrial (DJI) и S$\&$P 500 (GSPC). Этот новый статистический метод также указал на эффективность финансового рынка. Работа \cite{French1986} обнаружила небольшую автокорреляцию в дневных доходностях бирж NYSE и AMEX, однако она была настолько маленькой, что, по мнению авторов могла быть объяснена гипотезой о "торговом шуме". Она подразумевает, что трейдеры, видя операции друг друга, действуют нерационально и следуют друг за другом в надежде на то, что объект наблюдения владеет более полной информацией о цене акции. Другая работа, \cite{Lo1987}, используя недельные данные Исследовательского центра по оценке стоимости ценных бумаг\footnote{Center for Research in Security Prices, CSRP}, показала, что гипотеза о свободном блуждании неверна. Дальнейшие работы также отвергли гипотезу о свободном блуждании, при этом предположение о слабой эффективности рынка было подтверждено многочисленными исследованиями. Работа \cite{Taylor1982} предлагает новую версию теста на свободное блуждание, а также развивает идею о наличии тренда в ценах как объяснения слабой автокорреляции. Новые статистические методы были опробованы на ценах фьючерсов на 11 сырьевых товаров на бирже Великобритании. Результатом стало признание почти всех рынков не соответствующими модели случайного блуждания. Модель с трендом, с другой стороны, лучше сумела описать поведение данных.

Можно заметить, что разработка теоретической базы гипотезы об эффективности рынка шла преимущественно в США и данные брались именно по этому рынку. Большинство публикаций, проверяющих данную теорию для развивающихся рынков относятся ко второй половине 1990-х~гг. и началу 2000-х~гг. На тот момент слабая форма эффективности начала ассоциироваться не с соответствием процессу случайного блуждания, а со стационарностью временного ряда. В работе \cite{Mishra2017} перечислен ряд работ по развивающимся азиатским площадкам, индексы которых проверялись на наличие единичного корня. В этой работе также проверена гипотеза об эффективности фондовых индексов средних и малых индийских предприятий. В работе используются расширенный тест Дики-Фуллера, тест Филлипса-Перрона, тест KPSS~(Kwiatkowski, Phillips, Schmidt and Shin), а также тест на наличие единичного корня в моделях GARCH. Первые 4 теста показали наличие единичного корня, не отвергнув гипотезу об эффективности рынка. Последний тест показал возвращение к среднему в 3 индексах из 4. Один из индексов напоминает процесс случайного блуждания.

\cite{Wang2015} проверяет на стационарность фондовые индикаторы 7 регионов: Тайланда, Материкового Китая, Гонконга, Японии, Южной Кореи, Малайзии и Сингапура. Помимо уже перечисленных выше стандартных тестов~(Дики-Фуллера(расширенный), Филлипса-Перрона и KPSS), авторы используют более современные тесты: DF-GLS тест, модификацию теста Дики-Фуллера, описанную в \cite{Elliot1996}, и LM Fourier тест, описанный в \cite{Enders2012}. Результаты представлены в таблице \ref{tab:wang}.

\begin{table}[]
\centering
\caption{Результаты работы \cite{Wang2015}}
\label{tab:wang}
\begin{tabular}{|l|c|c|c|c|c|}
\hline
 & ADF & PP & KPSS & DF-GLS & LM-Fourier \\ \hline
Таиланд & Not rej & Not rej & Not rej & Not rej & Rej \\ \hline
Китай & Not rej & Rej & Rej & Not rej & Rej \\ \hline
Гонконг & Rej & Rej & Rej & Not rej & Rej \\ \hline
Япония & Not rej & Not rej & Not rej & Not rej & Rej \\ \hline
Южная Корея & Not rej & Not rej & Rej & Not rej & Rej \\ \hline
Малайзия & Not rej & Not rej & Rej & Rej & Rej \\ \hline
Сингапур & Not rej & Not rej & Rej & Not rej & Rej \\ \hline
\end{tabular}

{\raggedright \par $H_0$ для всех тестов, кроме LM-Fourier, гласит, что ряд стационарен.}

{\raggedright \par Для LM-Fourier $H_0$ гласит, что нет возврата к среднему.}
\end{table}

Итак, обзор литературы показывает, что идея об эффективности рынка сильно эволюционировала за последние 70 лет и повторные исследования одних и тех же данных с использованием различных методов давали различные результаты.

\newpage
\section{Данные и методология}\label{sec:meth}

В основу данного исследования легла работа \cite{Liu1997}. В ней проводится анализ китайского рынка акций на соответствии гипотезе о случайном блуждании и авторы приходят к выводу, что несмотря на низкий уровень развития китайского фондового рынка на тот момент, гипотеза выполняется. Liu et al использовали данные за 3 года работы Шанхайской и Шэньчжэньской бирж с 21~мая~1992~г. по 18~декабря~1995~г. Общее количество наблюдений составило 933 точки.

В данном исследовании используются цены закрытия индексов Шанхайской и Шэньчжэньской бирж. В случае биржи Шанхая рассматривались данные с 1~апреля~2007~г. по 1~февраля~2019~г. (2943 наблюдения), а в случае биржи Шэньчжэня -- с 14~сентября~2012~г. по 7~мая~2019~г.(1611 наблюдений). Точки отсечения были выбраны исходя из доступности данных и момента начала исследования.

Статистический тест был впервые описан в \cite{Dickey1981} и для целей данного исследования был реализован с помощью языка программирования R. Сначала были оценены параметры модели 1(уравнение \ref{eq:model1}).
\begin{equation}
  Y_t=\alpha+\rho Y_{t-1}+e_t\quad (t=2,3,\ldots,n),
\label{eq:model1}
\end{equation}
где $Y_1$ фиксирован и $e_t \sim \mathcal{N}(0,\,\sigma^{2})$. Параметры модели оцениваются с помощью уравнений \ref{eq:mod1estrho} и \ref{eq:mod1estalpha}. Они представляют собой МНК-оценку $\rho$ и $\alpha$. Для реализации в R исходные данные были преобразованы в объект типа zoo\footnote{Изначально этот пакет был написан для языка программирования S3, предшественника R. Zoo расшифровывается как "Z's Ordered Observations"\ , упорядоченные наблюдения переменной Z.}. Эта трансформация позволила использовать функцию lag из стандартного пакета stats, что упростило написание кода.
\begin{align}
  \label{eq:mod1estrho}
  \hat{\rho}&=\left[\sum_{t=2}^n (Y_{t-1}-\bar{y}_{(-1)})^2\right]^{-1}
  \sum_{t=2}^n \left(Y_{t-1}-\bar{y}_{(-1)}\right)\\
  \label{eq:mod1estalpha}
  \hat{\alpha}_\mu&=\bar{y}_{(0)}-\hat{\rho_\mu}\bar{y}_{(-1)},\\
where\quad
  \bar{y}_{(i)}&=(n-1)^{-1}\sum_{t=2}^{n}Y_{t+i}
\end{align}

T-статистика, которая показывает значимость этих параметров оценивается с помощью уравнения \ref{eq:mod1tstat}.

\begin{align}
  \label{eq:mod1tstat}
  \hat{\tau}_{\alpha\mu}=&S^{-1}_{\alpha\mu}\hat{\alpha}_\mu,\quad
  where\\
  S^2_{\alpha\mu}=&S^2_{e\mu}
  \left[ (n-1)^{-1}+\bar{y}^2_{(-1)}
  \left\{
  \sum_{t=2}^{n}(Y_{t-1}-\bar{y}_{(-1)})^2\right\}^{-1}\right],\\
  S^2_{e\mu}=&(n-3)^{-1}\sum_{t=2}^{n}(Y_t-\hat{\alpha_\mu}-\hat{\rho}_\mu Y_{t-1})^2.
\end{align}

Также оцениваются параметры альтернативной модели (уравнение \ref{eq:model2}).

\begin{equation}
  \label{eq:model2}
  Y_t=\alpha+\beta(t-1-0.5n)+\rho Y_{t-1}+e_t\quad (t=2,3,\ldots,n),
\end{equation}
где $Y_1$ фиксирован, а ошибки имеют нормальное распределение со средним 0 и дисперсией $\sigma^2$. Пусть $X$ -- матрица размерности $(n-1)\times 3$, ряд номера $i$ которой представлен $(1, i-0.5n, Y_i)$, а $Y'=(Y_2,Y_3,\ldots,Y_n)$. Тогда МНК оценка вектора параметров модели $\theta=(\alpha,\beta,\rho)$ представлена уравнением \ref{eq:mod2est}. Для написания кода в R использовался пакет matlib, который упрощает работу с матрицами, в частности, включает в себя команду для извлечения элементов диагонали матрицы.

\begin{equation}
  \label{eq:mod2est}
  \hat{\theta}_\tau=(\hat{\alpha}_\tau,\hat{\beta}_\tau,\hat{\rho}_\tau)'=(X'X)^{-1}X'Y
\end{equation}

Значимость параметров в этой модели определяется уравнениями \ref{mod2taualpha} и \ref{mod2taubeta}.
\begin{align}
  \label{mod2taualpha}
  \hat{\tau}_{\alpha\tau}&=(C_{11}S^2_{e\tau})^{-0.5}\hat{\alpha}_\tau,\\
  \label{mod2taubeta}
  \hat{\tau}_{\beta\tau}&=(C_{22}S^2_{e\tau})^{-0.5}\hat{\beta}_\tau,\quad where\\
  S^2_{e\tau}&=(n-4)^{-1}Y'\left[I-X(X'X)^{-1}X'\right]Y.
\end{align}
Далее авторы методологии приводят функцию правдоподобия для выборки, состоящей из $n$ наблюдений(уравнение \ref{eq:likefun}).

\begin{align}
\label{eq:likefun}
log{(L)}=-0.5(n-1)log(2\pi)-(n-1)log{(\sigma)}-(2\sigma^2)^{-1}\sum_{t=2}^{n}
\left[Y_t-\alpha-\beta(t-1-0.5n)-\rho Y_{t-1}\right]^2
\end{align}
В случае соответствия модели случайному блужданию с нулевым дрифтом ($H_0:(\alpha,\beta,\rho)=(0,0,1)$) при максимизации мы бы получили равенство \ref{eq:sigmah0}.
\begin{equation}
  \label{eq:sigmah0}
  \hat{\sigma}_0=(n-1)^{(-1)}\sum_{t=2}^n(Y_t-Y_{t-1})^2
\end{equation}
При отвержении нулевой гипотезы максимум правдоподобия был бы достигнут в точке $(\hat{\sigma}_1^2,\hat{\theta}_\tau^{'})$, где $\hat{\theta}_\tau$ расчитывается по формуле \ref{eq:mod2est} и $\hat{\sigma}_1^2=(n-4)(n-1)^{(-1)}S^2_{e\tau}$. Таким образом, отношение правдоподобия принимает вид уравнения \ref{eq:likeratio}.
\begin{align}
  \label{eq:likeratio}
  \left[\hat{\sigma}_0^{-1}\hat{\sigma}_1\right]^{n-1}&=\left[1+3(n-4)^{-1}\Phi_2\right],\quad where\\
  \Phi_2&=(3S^2_{e\tau})\left[(n-1)\hat{\sigma}_0^2-(n-4)S^2_{e\tau}\right]
\end{align}
Именно значение $\Phi_2$ определяет результат проверки гипотезы. Относительно него составляется стандартный F-test гипотезы $H_0:(\alpha,\beta,\rho)=(0,0,1)$. Для проверки гипотезы $H_0:(\alpha,\beta,\rho)=(\alpha,0,1)$ аналогичным значением является~$\Phi_3$, рассчитываемое по формуле~\ref{eq:phi3}.
\begin{equation}
  \label{eq:phi3}
  \Phi_3=(2S_{e\tau}^2)^{-1}
  \left[
    (n-1)
      \left\{
        \hat{\sigma}_0^2-(\bar{y}_{(0)}-\bar{y}_{(-1)})^2
      \right\}
      -(n-4)S^2_{e\tau}
  \right]
\end{equation}
Как можно заметить, $\Phi_2$ и $\Phi_3$ используются для проверки гипотез относительно второй модели (уравнение \ref{eq:model2}). Относительно первой модели (уравнение \ref{eq:model1}) проверяется гипотеза $H_0:(\alpha,\rho)=(0,1)$ с помощью показателя $\Phi_1$(уравнение \ref{eq:phi1}).
\begin{equation}
  \label{eq:phi1}
  \Phi_2=(3S^2_{e\tau})\left[(n-1)\hat{\sigma}_0^2-(n-4)S^2_{e\tau}\right]
\end{equation}

Критические значения t-статистик коэффициентов и F-теста также приведены в \cite{Dickey1981}.

Помимо этой методики в работе используются расширенный тест Дики-Фуллера, тест Филлипса-Перрона и тест KPSS. Они были реализованы в R в нескольких пакетах для работы с временными рядам. Для расчетов в этой работе используется пакет aTSA\footnote{Alternative Time Series Analysis, пакет разработан и поддерживается Дебином Цю. \\URL: https://cran.r-project.org/web/packages/aTSA/index.html}.

Расширенный тест Дики-Фуллера есть не что иное, как обобщение стандартного теста Дики-Фуллера для процесса авторегрессии порядка $p$. В рамках данного теста используется тот факт, что при наличии единственного единичного корня $z=1$ среди решений характеристического уравнения \ref{eq:chareq} полином порядка $p$ от лагового оператора равен нулю, то есть в модели, представленной уравнением \ref{eq:adf}, $\pi=0$ означает, что $\theta(1)=0$. Далее составляется t-статистика для $\pi$, которая сопоставляется с критическими значениями.

\begin{align}
  \label{eq:chareq}
  \theta(z)&=0,\\
  \theta(L)&=1-\theta_1 L-\theta_2 L^2-\ldots-\theta_p L^p
\end{align}

\begin{align}
  \label{eq:adf}
  \Delta y_t&=\pi y_{t-1}+c_1\Delta y_{t-1}+\ldots+c_{p-1}\Delta y_{t-p+1}+\varepsilon_t,\quad where\\
  \pi&=\theta_1+\ldots+\theta_p-1
\end{align}

Тест Филлипса-Перрона основан на тесте Дики-Фуллера, но корректирует полученные t-статистики, чтобы учесть потенциальную структуру автокорреляции остатков.

Тест KPSS устроен иначе. Идея состоит в том, чтобы разложить временной ряд по компонентам и проверить гипотезу о том, что дисперсия компоненты случайного блуждания равна 0.

Нет единого мнения о том, какой из этих тестов лучше и какой следует применять для разных типов выборки, поэтому чаще всего в литературе применяются все три теста на стационарность временного ряда.

\newpage
\section{Расчеты}\label{sec:quant}
При оценке первой модели на данных шанхайской биржи было получено уравнение \ref{eq:mod1res}. Значение t-статистики для коэффициента $\alpha$ равно $2.163$, что больше критического значения для бесконечного количества наблюдений при уровне значимости в 10\%. Соответственно, данный коэффициент значим.
  \begin{equation}
    \label{eq:mod1res}
    Y_t=0.023+0.997Y_{t-1}+e_t
  \end{equation}
Проверим гипотезу о соответствии цены закрытия индекса шанхайской фондовой биржи процессу случайного блуждания без дрифта.
\begin{align}
  H_0:&(\alpha,\rho)=(0,1)\\
  H_A:&(\alpha,\rho)\neq(0,1)\\
  \Phi_1&=1
\end{align}
В нашем случае значение $\Phi_1$ меньше критического при уровне значимости 10\%. Следовательно, нулевая гипотеза отвергается на этом и меньших уровнях значимости.

Следующим шагом является оценка альтернативной модели \ref{eq:model2}. Получаем уравнение вида \ref{eq:mod2res}. Для коэффициента $\alpha$ t-статистика составила $0.0001$, что меньше критического значения для бесконечного количества наблюдений на уровне значимости в 10\%. Данный коэффициент не значим. Для коэффициента $\beta$ t-test показал бесконечность. Следовательно, он значим на всех уровнях.
\begin{equation}
  \label{eq:mod2res}
  Y_t=0.02+0.000343(t-1-0.5n)+0.99Y_{t-1}+e_t
\end{equation}
Проверим гипотезу о том, что распределение цены закрытия шанхайской биржи соответствует процессу случайного блуждания без дрифта для данной модели.
\begin{align}
  H_0:&(\alpha, \beta, \rho)=(0,0,1)\\
  H_A:&(\alpha,\beta, \rho)\neq(0,0,1)\\
  \Phi_2&=-979.67
\end{align}
Тестовая статистика ушла в отрицательную область, что говорит о том, что нулевая гипотеза отвергается. Полученные показатели тестовых статистик говорят о том, что, скорее всего, оцененная модель имеет низкое качество и не может быть использована для получения осмысленных выводов для этой выборки. Тем не менее, попробуем проверить гипотезу о том, что данные все же следуют процессу случайного блуждания с неким неизвестным параметром дрифта $\alpha$.
\begin{align}
  H_0:&(\alpha, \beta, \rho)=(\alpha,0,1)\\
  H_A:&(\alpha,\beta, \rho)\neq(\alpha,0,1)\\
  \Phi_3&=-1469.502
\end{align}
Полученная тестовая статистика также приняла отрицательные значения. Следовательно, альтернативная модель также не позволяет говорить о соответствии индекса шанхайского фондового рынка гипотезе о случайном блуждании.

Повторим все эти операции для фондового индекса биржи Шэньчжэня. Оцененная первая модель приняла вид уравнения \ref{eq:mod1res2}. t-статистика для свободного члена равна $2.17$, что выше критического значения на уровне значимости $10\%$. Значит, коэффициент значим.
\begin{equation}
  \label{eq:mod1res2}
  Y_t=0.053+0.994Y_{t-1}+e_t
\end{equation}
Проверим гипотезу о случайном блуждании для первой модели.
\begin{align}
  H_0:&(\alpha,\rho)=(0,1)\\
  H_A:&(\alpha,\rho)\neq(0,1)\\
  \Phi_1&=1
\end{align}
Тестовая статистика вновь оказалась равна 1, что меньше критического значения и позволяет отвергнуть нулевую гипотезу.

Оценка альтернативной модели дала уравнение \ref{eq:mod2res2}. t-статистика для коэффициента $\alpha$ очень близка к нулю, что говорит о его статистической незначимости. t-статистика для $\beta$ показала бесконечность, что намекает на его значимость, но вызывает сомнения относительно качества модели.
\begin{equation}
  \label{eq:mod2res2}
  Y_t=0.053+0.0005(t-1-0.5n)+0.994Y_{t-1}+e_t
\end{equation}
Проведем тесты на соответствие модели случайного блуждания с и без дрифта.
\begin{align}
  H_0:&(\alpha, \beta, \rho)=(0,0,1)\\
  H_A:&(\alpha,\beta, \rho)\neq(0,0,1)\\
  \Phi_2&=-536\\
  H_0:&(\alpha, \beta, \rho)=(\alpha,0,1)\\
  H_A:&(\alpha,\beta, \rho)\neq(\alpha,0,1)\\
  \Phi_3&=-804
\end{align}
В этом случае тестовые статистики также приняли отрицательные значения.

Как было отмечено в разделе \ref{sec:litrev} существует также направление исследований, которое говорит о возможности проверки эффективности рынка путем определения стационарности временного ряда. Проведем ряд тестов в этом направлении.

Начнем с расширенного теста Дики-Фуллера в той форме, в которой он описан в \cite{fuller2009}. Нулевая гипотеза состоит в том, что во временном ряде присутствует единичный корень и ряд нестационарен. Альтернативная гипотеза предполагает стационарность ряда. Проведем тест на данных шанхайской биржи.

\begin{Shaded}
\begin{Highlighting}[]
\KeywordTok{adf.test}\NormalTok{(ds)}
\end{Highlighting}
\end{Shaded}
\begin{multicols}{2}
\begin{Verbatim}[fontsize=\small]

## Augmented Dickey-Fuller Test
## alternative: stationary
##
## Type 1: no drift no trend
##       lag      ADF p.value
##  [1,]   0 -0.01785   0.639
##  [2,]   1 -0.00484   0.643
##  [3,]   2  0.00119   0.644
##  [4,]   3 -0.00979   0.641
##  [5,]   4 -0.01997   0.638
##  [6,]   5 -0.02217   0.638
##  [7,]   6 -0.01323   0.640
##  [8,]   7 -0.01098   0.641
##  [9,]   8 -0.01031   0.641
## Type 2: with drift no trend
##       lag   ADF p.value
##  [1,]   0 -1.93   0.355
##  [2,]   1 -1.97   0.342
##  [3,]   2 -1.92   0.361
##  [4,]   3 -1.97   0.342
##  [5,]   4 -2.09   0.291
##  [6,]   5 -2.08   0.298
##  [7,]   6 -1.96   0.345
##  [8,]   7 -2.02   0.322
##  [9,]   8 -2.01   0.322
## Type 3: with drift and trend
##       lag   ADF p.value
##  [1,]   0 -1.92   0.611
##  [2,]   1 -1.95   0.597
##  [3,]   2 -1.90   0.618
##  [4,]   3 -1.95   0.597
##  [5,]   4 -2.08   0.542
##  [6,]   5 -2.06   0.549
##  [7,]   6 -1.94   0.600
##  [8,]   7 -2.00   0.575
##  [9,]   8 -2.00   0.576
## ----
## Note: in fact, p.value = 0.01
## means p.value <= 0.01
\end{Verbatim}
\end{multicols}
Как можно заметить по выдаче, расширенный тест проверяет данные на соответствие трем процессам: без тренда и дрифта, с дрифтом без тренда и с обоими. Если сравнить полученные значения с критическими, то $H_0$ не отвергается на уровне значимости в $5\%$ и ниже. Высокое значение p.value также указывает на то, что нулевая гипотеза не отвергается. Значит, временной ряд цены закрытия шанхайской биржи не является стационарным.

Проведем тот же тест для биржи Шэньчжэня. Результаты приведены ниже.

\begin{Shaded}
\begin{Highlighting}[]
\KeywordTok{adf.test}\NormalTok{(ds2)}
\end{Highlighting}
\end{Shaded}
\begin{multicols}{2}
\begin{Verbatim}[fontsize=\small]
## Augmented Dickey-Fuller Test
## alternative: stationary
##
## Type 1: no drift no trend
##      lag     ADF p.value
## [1,]   0 -0.0718   0.623
## [2,]   1 -0.1129   0.611
## [3,]   2 -0.1556   0.599
## [4,]   3 -0.1839   0.591
## [5,]   4 -0.1937   0.588
## [6,]   5 -0.2380   0.575
## [7,]   6 -0.2356   0.576
## [8,]   7 -0.2375   0.576
## Type 2: with drift no trend
##      lag   ADF p.value
## [1,]   0 -2.04   0.313
## [2,]   1 -1.99   0.330
## [3,]   2 -1.93   0.356
## [4,]   3 -2.01   0.326
## [5,]   4 -2.04   0.310
## [6,]   5 -2.06   0.306
## [7,]   6 -2.01   0.323
## [8,]   7 -2.20   0.246
## Type 3: with drift and trend
##      lag   ADF p.value
## [1,]   0 -2.20   0.494
## [2,]   1 -2.13   0.519
## [3,]   2 -2.05   0.554
## [4,]   3 -2.11   0.528
## [5,]   4 -2.15   0.514
## [6,]   5 -2.14   0.517
## [7,]   6 -2.10   0.534
## [8,]   7 -2.28   0.457
## ----
## Note: in fact, p.value = 0.01
## means p.value <= 0.01
\end{Verbatim}
\end{multicols}

В отношении биржи Шэньчжэня были получены аналогичные результаты. Следовательно, расширенный тест Дики-Фуллера показал, что временные ряды цен закрытия обеих бирж нестационарны.

Попробуем проверить их с помощью теста Филлипса-Перрона, также называемого непараметрическим тестом на наличие единичного корня. Нулевая и альтернативная гипотезы для данного теста совпадают с расширенным тестом Дики-Фуллера. Ниже приведены результаты данного теста для биржи Шанхая.

\begin{Shaded}
\begin{Highlighting}[]
\KeywordTok{pp.test}\NormalTok{(ds)}
\end{Highlighting}
\end{Shaded}

\begin{multicols}{2}
\begin{Verbatim}[fontsize=\small]
## Phillips-Perron Unit Root Test
## alternative: stationary
##
## Type 1: no drift no trend
##  lag    Z_rho p.value
##    9 -0.00246   0.692
## -----
##  Type 2: with drift no trend
##  lag Z_rho p.value
##    9 -7.96   0.293
## -----
##  Type 3: with drift and trend
##  lag Z_rho p.value
##    9 -7.93   0.588
## ---------------
## Note: p-value = 0.01 means
## p.value <= 0.01
\end{Verbatim}
\end{multicols}

Критические значения для теста Филлипса-Перрона совпадают с таковыми для расширенного тест Дики-Фуллера. Тест на соответствие моделям второго и третьего типа дал положительный результат, однако высокое значение p.value говорит о том, что оба показателя статистически незначимы. Тот факт, что они были получены на таком высоком лаге и были отобраны тестом как лучшие также позволяет усомниться в их статистической значимости.

Проведем аналогичный тест для данных биржи Шэньчжэня.

\begin{Shaded}
\begin{Highlighting}[]
\KeywordTok{pp.test}\NormalTok{(ds2)}
\end{Highlighting}
\end{Shaded}
\begin{multicols}{2}
\begin{Verbatim}[fontsize=\small]
## Phillips-Perron Unit Root Test
## alternative: stationary
##
## Type 1: no drift no trend
##  lag    Z_rho p.value
##    8 -0.00591   0.691
## -----
##  Type 2: with drift no trend
##  lag Z_rho p.value
##    8  -8.9   0.239
## -----
##  Type 3: with drift and trend
##  lag Z_rho p.value
##    8 -9.85   0.468
## ---------------
## Note: p-value = 0.01 means
## p.value <= 0.01
\end{Verbatim}
\end{multicols}

Результаты для второй крупнейшей биржи материкового Китая в точности повторили таковые для данных биржи Шанхая. Снова можно наблюдать высокий порядок лага, высокие значения p.value при самом значении тестовой статистики, указывающем на отвержение нулевой гипотезы.

Наконец, проведем третий классический тест на стационарность временного ряда, тест Kwiatkowski, Phillips, Schmidt, Shin или КПСС-тест. В этом тесте заложены обратные нулевые и альтернативные гипотезы: $H_0$ состоит в стационарности временного ряда, а $H_A$ -- в наличии единичного корня. Проведем данный тест для данных Шанхайской фондовой биржи.

\begin{Shaded}
\begin{Highlighting}[]
\KeywordTok{kpss.test}\NormalTok{(ds)}
\end{Highlighting}
\end{Shaded}

\begin{multicols}{2}\begin{Verbatim}[fontsize=\small]
## KPSS Unit Root Test
## alternative: nonstationary
##
## Type 1: no drift no trend
##  lag   stat p.value
##   12 0.0723     0.1
## -----
##  Type 2: with drift no trend
##  lag   stat p.value
##   12 0.0974     0.1
## -----
##  Type 1: with drift and trend
##  lag  stat p.value
##   12 0.099     0.1
## -----------
## Note: p.value = 0.01 means p.value <= 0.01
##     : p.value = 0.10 means p.value >= 0.10
\end{Verbatim}
\end{multicols}
Итак, мы видим, что нулевая гипотеза отвергается на уровне значимости $10\%$. Временной ряд не является стационарным. Проверим цены закрытия биржи Шэньчжэня.

\begin{Shaded}
\begin{Highlighting}[]
\KeywordTok{kpss.test}\NormalTok{(ds2)}
\end{Highlighting}
\end{Shaded}

\begin{multicols}{2}\begin{Verbatim}[fontsize=\small]
## KPSS Unit Root Test
## alternative: nonstationary
##
## Type 1: no drift no trend
##  lag   stat p.value
##    9 0.0701     0.1
## -----
##  Type 2: with drift no trend
##  lag stat p.value
##    9 0.16     0.1
## -----
##  Type 1: with drift and trend
##  lag   stat p.value
##    9 0.0728     0.1
## -----------
## Note: p.value = 0.01 means p.value <= 0.01
##     : p.value = 0.10 means p.value >= 0.10
\end{Verbatim}
\end{multicols}

В этом случае мы также получили нестационарность временного ряда на уровне значимости $10\%$.

Итак, результаты статистических тестов представлены в таблице \ref{tab:modres}. Китайские фондовые рынки не соответствуют гипотезе о случайном блуждании и гипотезе о слабой форме эффективности.

\begin{table}[]
\centering
\caption{Результаты статистических тестов исследования.}
\label{tab:modres}
\begin{tabular}{|c|l|c|c|}
\hline
\multicolumn{1}{|l|}{} & \multicolumn{1}{c|}{$H_0$} & Шанхай & Шэньчжэнь \\ \hline
$\Phi$ & Соответствие процессу случайного блуждания & Rej & Rej \\ \hline
ADF & Наличие единичного корня, нестационарность & Not rej(*) & Not rej(*) \\ \hline
PP & Наличие единичного корня, нестационарность & Rej & Rej \\ \hline
KPSS & Отсутствие единичного корня, стационарность & Rej(*) & Rej(*) \\ \hline
\end{tabular}
\end{table}
\newpage
\section{Интерпретация результатов и заключение}\label{sec:res}

Почему многочисленные исследования западных фондовых площадок говорят об эффективности рынка, а китайские биржи продемонстрировали неэффективность? Скорее всего, причина лежит в определенных особенностях китайской финансовой системы.

В литературе можно встретить следующее перечисление функций фондового рынка:
\begin{enumerate}
  \item Трансформация сбережений в инвестиции;
  \item Межотраслевой переток капитала через M\&A;
  \item Распространение экономической информации;
  \item Мониторинг эффективности предприятий;
  \item Стимулирование менеджмента предприятия.
\end{enumerate}

В странах с развитой финансовой системой и долгой историей развития фондового рынка все перечисленные функции выполняются. В Китае в силу новизны и особенностей государственного регулирования это не так.

Благодаря комбинации высокой нормы сбережения (40\%), государственного ограничения на доходность банковских депозитов (было отменено только в 2015~г.) и растущих располагаемых доходов населения, торговля ценными бумагами получила огромную популярность в Китае. Количество брокерских счетов сильно колебается год к году, медианное значение составило 137.25 миллионов счетов с 2012 по 2017 год, а максимальное количество за тот же период достигало 214.8 миллионов.

Межотраслевой переток капитала через сделки слияния и поглащения почти не осуществляется. В период с 1998 по 2009~гг. было заключено всего 364 сделки, из которых в 201 участвовали государственные компании. Из-за явления политического кредитования и важности политических связей частные фирмы в Китае часто ищут возможности для слияния с государственными предприятиями как способ улучшения своих финансовых показателей и получения возможности для развития бизнеса.

С точки зрения распространения финансовой информации наблюдается двоякая ситуация. В Китае существуют базовые требования по публикации финансовой отчетности для компаний-эмитентов, но они гораздо мягче, чем на ведущих зарубежных площадках. Если говорить о том, насколько финансовые индексы позволяют судить о реальном положении экономики, то эта функция не выполняется. Китайский фондовый рынок оторван от реального сектора экономики и фундаментальных показателей. Например, когда в 1999-2001~гг. в китайской экономике наблюдались застой и дефляция, индекс Шанхайской биржи продолжал расти.

Мониторинг эффективности предприятий и стимулирование менеджеров также не выполняются из-за сегментированности китайского фондового рынка. Так как китайское правительство рассматривало акционирование прежде всего как один из методов приватизации государственных предприятий, в основу создания акционерных обществ было заложено деление на блок акций, принадлежащих государству, блок госпредприятий, в отдельных случаях также блок персонала предприятий, и наконец блок физических лиц. Именно последний блок доступен для торговле на бирже, а остальные блоки сконцентрированы в руках государственных управляющих, людей близких к КПК и чиновников. У этой группы нет стимулов к тщательному мониторингу деятельности предприятий и контролю за менеджментом.

Многие работы подчеркивают спекулятивных характер фондового рынка в Китае, его склонность к строительству "воздушных замков \". На наш взгляд, именно в этом может скрываться причина неэффективности фондовых рынков, продемонстрированная в данном исследовании. Информационная функция цен на китайском фондовом рынке нарушена по причине его сегментации, особенностей регулирования и оторванности от фундаментальных факторов.

Отдельно следует сказать о результатах работы \cite{Liu1997}, которые во многом стали вдохновением для данного исследования. Возможно, что результат об эффективности китайского фондового рынка в этой работе был получен по причине небольшого количества наблюдений (рассматривались данные всего за 3.5 года), или несовершенств статистических тестов.

Такие особенности китайского фондового рынка открывают возможности для последующих исследований как в области гипотезы об эффективности рынка, так и событийного анализа.
\newpage
\bibliographystyle{apalike}
\bibliography{sources.bib}

\end{document}
