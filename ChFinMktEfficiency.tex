\documentclass[a4paper,10pt]{article}

\usepackage[T1,T2A]{fontenc}
\usepackage[utf8]{inputenc}
\usepackage[english,russian]{babel}
\usepackage{cmap}					% enable search in PDF

\usepackage{amsmath,amsfonts,amssymb,amsthm,mathtools} % AMS

\usepackage{geometry} % Setting margins

\geometry{top=20mm}

\geometry{bottom=20mm}

\geometry{left=20mm}

\geometry{right=20mm}

\setlength{\parindent}{0pt}


\title{Являются ли рынки акций КНР эффективными?}

\author{Туманянц Артемий}

\date{2019\\ВШЭ\\Москва}

\begin{document}
\maketitle
\section{Literature review}
В современных учебниках по финансам гипотеза об эффективности рынков преподается в законченной и лаконичной форме. Приводится самое широкое определение "рыночной эффективности", приведенное в \cite{Jensen1978}: рынок считается эффективным относительно информационного множества $\omega$, если ни один агент не может получить экономическую прибыль при помощи торгового правила, основанного на информационном множестве $\omega$. Экономической прибылью считается выручка за вычетом издержек, скорректированных на риск.
\newpage

\bibliographystyle{apalike}

\bibliography{sources.bib}

\end{document}
