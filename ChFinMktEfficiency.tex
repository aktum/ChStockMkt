\documentclass[a4paper,10pt]{article}

\usepackage[T1,T2A]{fontenc}
\usepackage[utf8]{inputenc}
\usepackage[english,russian]{babel}
\usepackage{cmap}					% enable search in PDF

\usepackage{amsmath,amsfonts,amssymb,amsthm,mathtools} % AMS

\usepackage{geometry} % Setting margins

\geometry{top=20mm}

\geometry{bottom=20mm}

\geometry{left=20mm}

\geometry{right=20mm}

\setlength{\parindent}{0pt}


\title{Являются ли рынки акций КНР эффективными?}

\author{Туманянц Артемий}

\date{2019\\ВШЭ\\Москва}

\begin{document}
\maketitle
%Объем работы: минимум 15 страниц.
%Шрифт 12, интервал 1, выравнивание: по ширине
%Предполагается, что эта работа - Ваше авторское исследование выбранной проблематики:
%- введением на 1-2 стр. (актуальность, степень научной проработанности, объект, предмет, цель, задачи, методология расчетов, практическая значимость),
%- основной частью из нескольких глав (анализ и систематизация имеющихся научных работ (теоретическая база), описание выборки и методологии, расчетная часть),
%- заключением на 1-2 стр. (анализ и интерпретация полученных результатов, направления дальнейших исследований),
%- списком литературы,
%- приложением (если требуется).
\newpage
\tableofcontents
\newpage
\section{Введение}
Во многих развитых странах фондовый рынок и возможности, которые он предлагает для частных лиц, уже давно стали частью житейской мудрости. По данным опроса Gallup в 2016~г. 52\%  американцев инвестировали свои средства в рынок акций. Для них торгуемые финансовые инструменты являются одной из форм сбережений. Большинство понимает, что эти инвестиции значительно более рискованны, чем условный вклад в банке, но они готовы пойти на риск ради более высокой доходности. Популярные книги по инвестированию предупреждают инвесторов о том, что американский рынок акций "эффективен", то есть получение сверхдоходности является скорее исключением, чем правилом. Среднестатистический индивидуальный инвестор, скорее всего, получит среднюю доходность по рынку, но не более того. Именно поэтому получила развитие сфера сберегательных планов, таких как 401(k), сберегательный счет, на который работодатель может перечислять часть доходов сотрудника. Эти сбережения передаются профессиональным управляющим активами, которые инвестируют их в паи, акции и облигации.

В развивающихся странах, например, в Китае ситуация иная. Рынок является новым явлением,

 Однако достоверного метода предсказания цены того или иного актива до сих пор не существует. Многие покупают акции в надежде, что их цена просто будет расти в следствии стадии экономического цикла, в котором находится финансовый рынок. В современной науке преобладает мнение, что рынок "эффективен", то есть получение сверхдоходов на рынке акций невозможно без использования инсайдерской информации или других фонм

\newpage
\bibliographystyle{apalike}
\bibliography{sources.bib}

\end{document}
