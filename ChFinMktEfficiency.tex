\documentclass[a4paper,12pt]{article}

\usepackage[T1,T2A]{fontenc} % make cyrillic symbols work
\usepackage[utf8]{inputenc}
\usepackage[english,russian]{babel}
\usepackage{cmap}					% enable search in PDF

\usepackage{amsmath,amsfonts,amssymb,amsthm,mathtools} % AMS

\usepackage{chngcntr}  % restart equation numbering in each section, subsection
\counterwithin*{equation}{section}
\counterwithin*{equation}{subsection}

\usepackage{geometry} % Setting margins

\usepackage{booktabs}

\geometry{top=20mm}

\geometry{bottom=20mm}

\geometry{left=20mm}

\geometry{right=20mm}

\newtheorem{definition}{Определение}

\setlength{\parindent}{1.27 cm} % start each paragraph with an indent

\usepackage{setspace} % spacing options
\singlespacing


\title{Являются ли рынки акций КНР эффективными?}

\author{Туманянц Артемий}

\date{2019\\ВШЭ\\Москва}

\begin{document}
%\maketitle

%Объем работы: минимум 15 страниц.
%Шрифт 12, интервал 1, выравнивание: по ширине
%Предполагается, что эта работа - Ваше авторское исследование выбранной проблематики:
%- введением на 1-2 стр. (актуальность, степень научной проработанности, объект, предмет, цель, задачи, методология расчетов, практическая значимость),
%- основной частью из нескольких глав (анализ и систематизация имеющихся научных работ (теоретическая база), описание выборки и методологии, расчетная часть),
%- заключением на 1-2 стр. (анализ и интерпретация полученных результатов, направления дальнейших исследований),
%- списком литературы,
%- приложением (если требуется).

%\newpage
%\tableofcontents
%\newpage
\section{Введение}

Актуальность:

Степень научной проработанности:

Объект: китайский биржевой рынок акций, представленный Шэньчжэньской и Шанхайской биржами

Предмет: эффективность китайского рынка акций

Цель: провести тест на соответствие динамики индексов Шэньчжэньской и Шанхайской бирж гипотезе о "случайном блуждании" котировок

Задачи: определить оптимальную процедуру теста, сравнить полученные результаты с ранее полученными показателями в чужих работах

Методология расчетов: статистика отношения правдоподобия для авторегрессионных рядов с единичным корнем

Практическая значимость: ???
\newpage
\section{Теоретическая база}

В данной работе проводится тест на выполнение гипотезы о "случайном блуждании", которая была разработана в 1950-1960-е годы и легла в основу гипотезы об эффективности рынка. \cite{Jensen1978} дает следующее определение эффективности рынка:

\begin{definition}
  Рынок считается эффективным по отношению к информационному множеству $\Omega$, если стратегия торговли, основанная на $\Omega$ не позволяет получать экономическую прибыль, т.е. положительную чистую выручку, скорректированную на риск.
\end{definition}

Выделяют три формы эффективности рынка в зависимости от содержания информационного множества $\Omega$:
\begin{enumerate}
  \item Слабая форма эффективности: в $\Omega$ заложены исторические данные об уровнях цен;
  \item Средняя форма эффективности: в $\Omega$ содержится вся публично доступная информация;
  \item Сильная форма эффективности: в $\Omega$ включены все данные известные кому-либо.
\end{enumerate}

Для целей данной работы наибольший интерес представляет слабая форма эффективности. \cite{Kendall1953} исследовал корреляцию между первыми разностями стоимости акций 19 компаний~(486 наблюдений) и 3 сырьевых индексов~(2387, 548 и 1446 наблюдений). Один из выводов анализа гласит, что данные ведут себя почти как блуждающие временные ряды. \cite{Granger1963} использовали спектральный анализ для проверки гипотезы о "случайном блуждании" на различных индексах финансовых рынков США, включая Dow-Jones Industrial (DJI) и S$\&$P 500 (GSPC). Этот новый статистический метод также указал на эффективность финансового рынка. Работа \cite{French1986} обнаружила небольшую автокорреляцию в дневных доходностях бирж NYSE и AMEX, однако она была настолько маленькой, что, по мнению авторов могла быть объяснена гипотезой о "торговом шуме". Она подразумевает, что трейдеры, видя операции друг друга, действуют нерационально и следуют друг за другом в надежде на то, что объект наблюдения владеет более полной информацией о цене акции. Другая работа, \cite{Lo1987}, используя недельные данные Исследовательского центра по оценке стоимости ценных бумаг\footnote{Center for Research in Security Prices, CSRP}, показала, что гипотеза о свободном блуждании неверна. Дальнейшие работы также отвергли гипотезу о свободном блуждании, при этом предположение о слабой эффективности рынка было подтверждено многочисленными исследованиями. Работа \cite{Taylor1982} предлагает новую версию теста на свободное блуждание, а также развивает идею о наличии тренда в ценах как объяснения слабой автокорреляции. Новые статистические методы были опробованы на ценах фьючерсов на 11 сырьевых товаров на бирже Великобритании. Результатом стало признание почти всех рынков не соответствующими модели случайного блуждания. Модель с трендом, с другой стороны, лучше сумела описать поведение данных.

Можно заметить, что разработка теоретической базы гипотезы об эффективности рынка шла преимущественно в США и данные брались именно по этому рынку. Большинство публикаций, проверяющих данную теорию для развивающихся рынков относятся ко второй половине 1990-х~гг. и началу 2000-х~гг. На тот момент слабая форма эффективности начала ассоциироваться не с соответствием процессу случайного блуждания, а со стационарностью временного ряда. В работе \cite{Mishra2017} перечислен ряд работ по развивающимся азиатским площадкам, индексы которых проверялись на наличие единичного корня. В этой работе также проверена гипотеза об эффективности фондовых индексов средних и малых индийских предприятий. В работе используются расширенный тест Дики-Фуллера, тест Филлипса-Перрона, тест KPSS~(Kwiatkowski, Phillips, Schmidt and Shin), а также тест на наличие единичного корня в моделях GARCH. Первые 4 теста показали наличие единичного корня, подтвердив выполнение гипотезы об эффективности рынка. Последний тест показал возвращение к среднему в 3 индексах из 4. Один из индексов напоминает процесс случайного блуждания.

\cite{Wang2015} проверяет на стационарность фондовые индикаторы 7 регионов: Тайланда, Материкового Китая, Гонконга, Японии, Южной Кореи, Малайзии и Сингапура. Помимо уже перечисленных выше стандартных тестов~(Дики-Фуллера(расширенный), Филлипса-Перрона и KPSS), авторы используют более современные тесты: DF-GLS тест, модификацию теста Дики-Фуллера, описанную в \cite{Elliot1996}, и LM Fourier тест, описанный в \cite{Enders2012}. Результаты представлены в таблице \ref{tab:wang}.

\begin{table}[]
\centering
\caption{Результаты работы \cite{Wang2015}}
\label{tab:wang}
\begin{tabular}{|l|c|c|c|c|c|}
\hline
 & ADF & PP & KPSS & DF-GLS & LM-Fourier \\ \hline
Таиланд & Not rej & Not rej & Not rej & Not rej & Rej \\ \hline
Китай & Not rej & Rej & Rej & Not rej & Rej \\ \hline
Гонконг & Rej & Rej & Rej & Not rej & Rej \\ \hline
Япония & Not rej & Not rej & Not rej & Not rej & Rej \\ \hline
Южная Корея & Not rej & Not rej & Rej & Not rej & Rej \\ \hline
Малайзия & Not rej & Not rej & Rej & Rej & Rej \\ \hline
Сингапур & Not rej & Not rej & Rej & Not rej & Rej \\ \hline
\end{tabular}

{\raggedright \par $H_0$ для всех тестов, кроме LM-Fourier, гласит, что ряд стационарен.}

{\raggedright \par Для LM-Fourier $H_0$ гласит, что нет возврата к среднему.}
\end{table}

Итак, обзор литературы показывает, что идея об эффективности рынка сильно эволюционировала за последние 70 лет и повторные исследования одних и тех же данных с использованием различных методов давали различные результаты.

\newpage
\section{Данные и методология}

В основу данного исследования легла работа \cite{Liu1997}. В ней проводится анализ китайского рынка акций на соответствии гипотезе о случайном блуждании и авторы приходят к выводу, что несмотря на низкий уровень развития китайского фондового рынка на тот момент, гипотеза выполняется. Liu et al использовали данные за 3 года работы Шанхайской и Шэньчжэньской бирж с 21~мая~1992~г. по 18~декабря~1995~г. Общее количество наблюдений составило 933 точки.

В данном исследовании используются цены закрытия индексов Шанхайской и Шэньчжэньской бирж. В случае биржи Шанхая рассматривались данные с 1~апреля~2007~г. по 1~февраля~2019~г. (2943 наблюдения), а в случае биржи Шэньчжэня -- с 14~сентября~2012~г. по 7~мая~2019~г.(1611 наблюдений). Точки отсечения были выбраны исходя из доступности данных и момента начала исследования.

Статистический тест был впервые описан в \cite{Dickey1981} и для целей данного исследования был реализован с помощью языка программирования R. Сначала были оценены параметры модели 1(уравнение \ref{eq:model1}).
\begin{equation}
  Y_t=\alpha+\rho Y_{t-1}+e_t\quad (t=2,3,\ldots,n),
\label{eq:model1}
\end{equation}
где $Y_1$ фиксирован и $e_t \sim \mathcal{N}(0,\,\sigma^{2})$. Параметры модели оцениваются с помощью уравнений \ref{eq:mod1estrho} и \ref{eq:mod1estalpha}. Они представляют собой МНК-оценку $\rho$ и $\alpha$. Для реализации в R исходные данные были преобразованы в объект типа zoo\footnote{Изначально этот пакет был написан для языка программирования S3, предшественника R. Zoo расшифровывается как "Z's Ordered Observations"\ , упорядоченные наблюдения переменной Z.}. Эта трансформация позволила использовать функцию lag из стандартного пакета stats, что упростило написание кода.
\begin{align}
%\begin{equation}
  \label{eq:mod1estrho}
  \hat{\rho}&=\left[\sum_{t=2}^n (Y_{t-1}-\bar{y}_{(-1)})^2\right]^{-1}
  \sum_{t=2}^n \left(Y_{t-1}-\bar{y}_{(-1)}\right)\\
%\end{equation}
%\begin{equation}
  \label{eq:mod1estalpha}
  \hat{\alpha}_\mu&=\bar{y}_{(0)}-\hat{\rho_\mu}\bar{y}_{(-1)},\\
%\end{equation}
where\quad
%\begin{equation}
  \bar{y}_{(i)}&=(n-1)^{-1}\sum_{t=2}^{n}Y_{t+i}
%\end{equation}
\end{align}

T-статистика, которая показывает значимость этих параметров оценивается с помощью уравнения \ref{eq:mod1tstat}.

\begin{align}
  \label{eq:mod1tstat}
  \hat{\tau}_{\alpha\mu}=&S^{-1}_{\alpha\mu}\hat{\alpha}_\mu,\quad
  where\\
  S^2_{\alpha\mu}=&S^2_{e\mu}
  \left[ (n-1)^{-1}+\bar{y}^2_{(-1)}
  \left\{
  \sum_{t=2}^{n}(Y_{t-1}-\bar{y}_{(-1)})^2\right\}^{-1}\right],\\
  S^2_{e\mu}=&(n-3)^{-1}\sum_{t=2}^{n}(Y_t-\hat{\alpha_\mu}-\hat{\rho}_\mu Y_{t-1})^2.
\end{align}

Также оцениваются параметры альтернативной модели (уравнение \ref{eq:model2}).

\begin{equation}
  \label{eq:model2}
  Y_t=\alpha+\beta(t-1-0.5n)+\rho Y_{t-1}+e_t\quad (t=2,3,\ldots,n),
\end{equation}
где $Y_1$ фиксирован, а ошибки имеют нормальное распределение со средним 0 и дисперсией $\sigma^2$. Пусть $X$ -- матрица размерности $(n-1)\times 3$, ряд номера $i$ которой представлен $(1, i-0.5n, Y_i)$, а $Y'=(Y_2,Y_3,\ldots,Y_n)$. Тогда МНК оценка вектора параметров модели $\theta=(\alpha,\beta,\rho)$ представлена уравнением \ref{eq:mod2est}. Для написания кода в R использовался пакет matlib, который упрощает работу с матрицами, в частности, включает в себя команду для извлечения элементов диагонали матрицы.

\begin{equation}
  \label{eq:mod2est}
  \hat{\theta}_\tau=(\hat{\alpha}_\tau,\hat{\beta}_\tau,\hat{\rho}_\tau)'=(X'X)^{-1}X'Y
\end{equation}

Значимость параметров в этой модели определяется уравнениями \ref{mod2taualpha} и \ref{mod2taubeta}.
\begin{align}
  \label{mod2taualpha}
  \hat{\tau}_{\alpha\tau}&=(C_{11}S^2_{e\tau})^{-0.5}\hat{\alpha}_\tau,\\
  \label{mod2taubeta}
  \hat{\tau}_{\beta\tau}&=(C_{22}S^2_{e\tau})^{-0.5}\hat{\beta}_\tau,\quad where\\
  S^2_{e\tau}&=(n-4)^{-1}Y'\left[I-X(X'X)^{-1}X'\right]Y.
\end{align}
Далее авторы методологии приводят функцию правдоподобия для выборки, состоящей из $n$ наблюдений(уравнение \ref{eq:likefun}).

\begin{align}
\label{eq:likefun}
log{(L)}=-0.5(n-1)log(2\pi)-(n-1)log{(\sigma)}-(2\sigma^2)^{-1}\sum_{t=2}^{n}
\left[Y_t-\alpha-\beta(t-1-0.5n)-\rho Y_{t-1}\right]^2
\end{align}
В случае соответствия модели случайному блужданию с нулевым дрифтом ($H_0:(\alpha,\beta,\rho)=(0,0,1)$) при максимизации мы бы получили равенство \ref{eq:sigmah0}.
\begin{equation}
  \label{eq:sigmah0}
  \hat{\sigma}_0=(n-1)^{(-1)}\sum_{t=2}^n(Y_t-Y_{t-1})^2
\end{equation}
При отвержении нулевой гипотезы максимум правдоподобия был бы достигнут в точке $(\hat{\sigma}_1^2,\hat{\theta}_\tau^{'})$, где $\hat{\theta}_\tau$ расчитывается по формуле \ref{eq:mod2est} и $\hat{\sigma}_1^2=(n-4)(n-1)^{(-1)}S^2_{e\tau}$. Таким образом, отношение правдоподобия принимает вид уравнения \ref{eq:likeratio}.
\begin{align}
  \label{eq:likeratio}
  \left[\hat{\sigma}_0^{-1}\hat{\sigma}_1\right]^{n-1}&=\left[1+3(n-4)^{-1}\Phi_2\right],\quad where\\
  \Phi_2&=(3S^2_{e\tau})\left[(n-1)\hat{\sigma}_0^2-(n-4)S^2_{e\tau}\right]
\end{align}
Именно значение $\Phi_2$ определяет результат проверки гипотезы. Относительно него составляется стандартный F-test гипотезы $H_0:(\alpha,\beta,\rho)=(0,0,1)$. Для проверки гипотезы $H_0:(\alpha,\beta,\rho)=(\alpha,0,1)$ аналогичным значением является~$\Phi_3$, рассчитываемое по формуле~\ref{eq:phi3}.
\begin{equation}
  \label{eq:phi3}
  \Phi_3=(2S_{e\tau}^2)^{-1}
  \left[
    (n-1)
      \left\{
        \hat{\sigma}_0^2-(\bar{y}_{(0)}-\bar{y}_{(-1)})^2
      \right\}
      -(n-4)S^2_{e\tau}
  \right]
\end{equation}
Как можно заметить, $\Phi_2$ и $\Phi_3$ используются для проверки гипотез относительно второй модели (уравнение \ref{eq:model2}). Относительно первой модели (уравнение \ref{eq:model1}) проверяется гипотеза $H_0:(\alpha,\rho)=(0,1)$ с помощью показателя $\Phi_1$(уравнение \ref{eq:phi1}).
\begin{equation}
  \label{eq:phi1}
  \Phi_2=(3S^2_{e\tau})\left[(n-1)\hat{\sigma}_0^2-(n-4)S^2_{e\tau}\right]
\end{equation}

Критические значения t-статистик коэффициентов и F-теста также приведены в \cite{Dickey1981}.

\newpage
\section{Результаты расчетов}
При оценке первой модели на данных шанхайской биржи было получено уравнение \ref{eq:mod1res}. Значение t-статистики для коэффициента $\alpha$ равно $2.163$, что больше критического значения для бесконечного количества наблюдений при уровне значимости в 10\%. Соответственно, данный коэффициент значим.
  \begin{equation}
    \label{eq:mod1res}
    Y_t=0.023+0.997Y_{t-1}+e_t
  \end{equation}
Проверим гипотезу о соответствии цены закрытия индекса шанхайской фондовой биржи процессу случайного блуждания без дрифта.
\begin{align}
  H_0:&(\alpha,\rho)=(0,1)\\
  H_A:&(\alpha,\rho)\neq(0,1)\\
  \Phi_1&=1
\end{align}
В нашем случае значение $\Phi_1$ меньше критического при уровне значимости 10\%. Следовательно, нулевая гипотеза отвергается на этом и меньших уровнях значимости.

Следующим шагом является оценка альтернативной модели \ref{eq:model2}. Получаем уравнение вида \ref{eq:mod2res}. Для коэффициента $\alpha$ t-статистика составила $0.0001$, что меньше критического значения для бесконечного количества наблюдений на уровне значимости в 10\%. Данный коэффициент не значим. Для коэффициента $\beta$ t-test показал бесконечность. Следовательно, он значим на всех уровнях.
\begin{equation}
  \label{eq:mod2res}
  Y_t=0.02+0.000343(t-1-0.5n)+0.99Y_{t-1}+e_t
\end{equation}
Проверим гипотезу о том, что распределение цены закрытия шанхайской биржи соответствует процессу случайного блуждания без дрифта для данной модели.
\begin{align}
  H_0:&(\alpha, \beta, \rho)=(0,0,1)\\
  H_A:&(\alpha,\beta, \rho)\neq(0,0,1)\\
  \Phi_2&=-979.67
\end{align}
Тестовая статистика ушла в отрицательную область, что говорит о том, что нулевая гипотеза отвергается. Полученные показатели тестовых статистик говорят о том, что, скорее всего, оцененная модель имеет низкое качество и не может быть использована для получения осмысленных выводов для этой выборки. Тем не менее, попробуем проверить гипотезу о том, что данные все же следуют процессу случайного блуждания с неким неизвестным параметром дрифта $\alpha$.
\begin{align}
  H_0:&(\alpha, \beta, \rho)=(\alpha,0,1)\\
  H_A:&(\alpha,\beta, \rho)\neq(\alpha,0,1)\\
  \Phi_3&=-1469.502
\end{align}
Полученная тестовая статистика также приняла отрицательные значения. Следовательно, альтернативная модель также не позволяет говорить о соответствии индекса шанхайского фондового рынка гипотезе о случайном блуждании.

Повторим все эти операции для фондового индекса биржи Шэньчжэня. Оцененная первая модель приняла вид уравнения \ref{eq:mod1res2}. t-статистика для свободного члена равна $2.17$, что выше критического значения на уровне значимости $10\%$. Значит, коэффициент значим.
\begin{equation}
  \label{eq:mod1res2}
  Y_t=0.053+0.994Y_{t-1}+e_t
\end{equation}
Проверим гипотезу о случайном блуждании для первой модели.
\begin{align}
  H_0:&(\alpha,\rho)=(0,1)\\
  H_A:&(\alpha,\rho)\neq(0,1)\\
  \Phi_1&=1
\end{align}
Тестовая статистика вновь оказалась равна 1, что меньше критического значения и позволяет отвергнуть нулевую гипотезу.

Оценка альтернативной модели дала уравнение \ref{eq:mod2res2}. t-статистика для коэффициента $\alpha$ очень близка к нулю, что говорит о его статистической незначимости. t-статистика для $\beta$ показала бесконечность, что намекает на его значимость, но вызывает сомнения относительно качества модели.
\begin{equation}
  \label{eq:mod2res2}
  Y_t=0.053+0.0005(t-1-0.5n)+0.994Y_{t-1}+e_t
\end{equation}
Проведем тесты на соответствие модели случайного блуждания с и без дрифта.
\begin{align}
  H_0:&(\alpha, \beta, \rho)=(0,0,1)\\
  H_A:&(\alpha,\beta, \rho)\neq(0,0,1)\\
  \Phi_2&=-536\\
  H_0:&(\alpha, \beta, \rho)=(\alpha,0,1)\\
  H_A:&(\alpha,\beta, \rho)\neq(\alpha,0,1)\\
  \Phi_3&=-804
\end{align}
В этом случае тестовые статистики также приняли отрицательные значения.
\newpage
\bibliographystyle{apalike}
\bibliography{sources.bib}

\end{document}
