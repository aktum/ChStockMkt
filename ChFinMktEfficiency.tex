\documentclass[a4paper,12pt]{article}

\usepackage[T1,T2A]{fontenc}
\usepackage[utf8]{inputenc}
\usepackage[english,russian]{babel}
\usepackage{cmap}					% enable search in PDF

\usepackage{amsmath,amsfonts,amssymb,amsthm,mathtools} % AMS

\usepackage{geometry} % Setting margins

\geometry{top=20mm}

\geometry{bottom=20mm}

\geometry{left=20mm}

\geometry{right=20mm}

\setlength{\parindent}{0pt}

\usepackage{setspace} % Интерлиньяж
\singlespacing % Интерлиньяж 1

\newtheorem{definition}{Определение}

\title{Являются ли рынки акций КНР эффективными?}

\author{Туманянц Артемий}

\date{2019\\ВШЭ\\Москва}

\begin{document}
\maketitle
%Объем работы: минимум 15 страниц.
%Шрифт 12, интервал 1, выравнивание: по ширине
%Предполагается, что эта работа - Ваше авторское исследование выбранной проблематики:
%- введением на 1-2 стр. (актуальность, степень научной проработанности, объект, предмет, цель, задачи, методология расчетов, практическая значимость),
%- основной частью из нескольких глав (анализ и систематизация имеющихся научных работ (теоретическая база), описание выборки и методологии, расчетная часть),
%- заключением на 1-2 стр. (анализ и интерпретация полученных результатов, направления дальнейших исследований),
%- списком литературы,
%- приложением (если требуется).
\newpage
\tableofcontents
\newpage
\section{Введение}
Актуальность:

Степень научной проработанности:

Объект: китайский биржевой рынок акций, представленный Шэнчжэнской и Шанхайской биржами

Предмет: эффективность китайского рынка акций

Цель: провести тест на соответствие динамики индексов Шэнчжэнской и Шанхайской бирж гипотезе о "случайном блуждании" котировок

Задачи: определить оптимальную процедуру теста, сравнить полученные результаты с ранее полученными показателями в чужих работах

Методология расчетов: статистика отношения правдоподобия для авторегрессионных рядов с единичным корнем

Практическая значимость: ???
\section{Теоретическая база}
В данной работе проводится тест на выполнение гипотезы о "случайном блуждании", которая была разработана в 1950-1960-е годы и легла в основу гипотезы об эффективности рынка. \cite{Jensen1978} дает следующее определение эффективности рынка:
\begin{definition}
  \label{EM}
  Рынок считается эффективным по отношению к информационному множеству $\Omega$, если стратегия торговли, основанная на $\Omega$ не позволяет получать экономическую прибыль, т.е. положительную чистую выручку, скорректированную на риск.
\end{definition}
Выделяют три формы эффективности рынка в зависимости от содержания инфмормационного множества $\Omega$:
\begin{enumerate}
  \item Слабая форма эффективности: в $\Omega$ заложены исторические данные об уровнях цен;
  \item Средняя форма эффективности: в $\Omega$ содержится вся публично доступная информация;
  \item Сильная форма эффективности: в $\Omega$ включены все данные известные кому-либо.
\end{enumerate}
Для целей данной работы наибольший интерес представляет слабая форма эффективности. \cite{Peon2019} предлагает масштабный обзор работ, посвященных тестам именно этой формы. Их можно разделить на три крупные группы: (1)тесты на статистическую независимость цен или доходностей; (2) тесты торговых правил; (3) тесты на сезонность. К первой группе относятся ранние работы, в которых проводились тесты на присутствие автокорреляции во временных рядах. \cite{Kendall1953} исследовал корреляцию между первыми разностями стоимости акций 19 компаний~(486 наблюдений) и 3 сырьевых индексов~(2387, 548 и 1446 наблюдений). Одним из выводов анализа гласит, что данные ведут себя почти как блуждающие временные ряды. \cite{Granger1963} использовал спектральный анализ для проверки гипотезы о "случайном блуждании" на различных индексах финансовых рынков США, включая Dow-Jones Industrial (DJI) и S$\&$P 500 (GSPC). Этот новый статистический метод также указал на эффективность финансового рынка.
\newpage
\bibliographystyle{apalike}
\bibliography{sources.bib}

\end{document}
