%!TEX TS-program = xelatex
\documentclass{beamer}

\usepackage{HSE-theme/beamerthemeHSE} % Подгружаем тему

\usepackage[T1,T2A]{fontenc} % make cyrillic symbols work
\usepackage[utf8]{inputenc}
\usepackage[english,russian]{babel}
\usepackage{cmap}					% enable search in PDF

\usepackage{ulem}

%%% Работа с русским языком и шрифтами
\usepackage[english,russian]{babel}   % загружает пакет многоязыковой вёрстки
\usepackage{fontspec}      % подготавливает загрузку шрифтов Open Type, True Type и др.
\defaultfontfeatures{Ligatures={TeX},Renderer=Basic}  % свойства шрифтов по умолчанию
\setmainfont[Ligatures={TeX,Historic}]{Myriad Pro} %  установите шрифты Myriad Pro или (при невозможности) замените здесь на другой шрифт, который есть в системе — например, Arial
\setsansfont{Myriad Pro}  %  установите шрифты Myriad Pro или (при невозможности) замените здесь на другой шрифт, который есть в системе — например, Arial
\setmonofont{Courier New}
\uselanguage{russian}
\languagepath{russian}
\deftranslation[to=russian]{Theorem}{Теорема}
\deftranslation[to=russian]{Definition}{Определение}
\deftranslation[to=russian]{Definitions}{Определения}
\deftranslation[to=russian]{Corollary}{Следствие}
\deftranslation[to=russian]{Fact}{Факт}
\deftranslation[to=russian]{Example}{Пример}
\deftranslation[to=russian]{Examples}{Примеры}

\usepackage{multicol} 		% Несколько колонок
\graphicspath{{images/}}  	% Папка с картинками

%%% Информация об авторе и выступлении
\title[Являются ли фондовые рынки КНР эффективными?]{Являются ли фондовые рынки КНР эффективными?}
\subtitle{Актуальные проблемы современных фондовых рынков}
\author[Туманянц А.К.]{Туманянц А.К. \\ \smallskip \scriptsize \url{aktumanyants@edu.hse.ru}\\\url{https://www.hse.ru/org/persons/190919743}}
\institute[Высшая школа экономики]{Национальный исследовательский университет \\ «Высшая школа экономики» (Москва)}
\date{\today}

\begin{document}	% Начало презентации

\frame[plain]{\titlepage}	% Титульный слайд

\section{}
\subsection{}

\begin{frame}
\frametitle{Цель и задачи исследования}
\emph{Цель:} проверить динамику индексов Шэньчжэньской и Шанхайской бирж на соответствие гипотезам о случайном блуждании и о слабой эффективности рынка.

Задачи:
\begin{enumerate}
	\item Проанализировать теоретическую основу исследований об эффективности рынка;
  \item Провести тест на соответствие китайского фондового рынка гипотезе о случайном блужданнии;
  \item Провести ряд тестов на стационарность временного ряда цен закрытия фондовых бирж Китая;
  \item Сравнить полученные результаты с работами других исследователей.
\end{enumerate}
\end{frame}

\begin{frame}
\frametitle{Теоретическая основа}
\framesubtitle{От свободного блуждания к стационарности и дальше}
Путь развития гипотезы об эффективности рынка:
\begin{itemize}
	\item 1950-1970е: Гипотеза о свободном блуждании
	\item 1970е: Законченная форма гипотезы о трех формах эффективности рынка
	\item 1980е: Гипотезы о стационарности, о чрезмерной реакции рынка, об оптимизме инвесторов и тд.
\end{itemize}
Результаты исследований фондового рынка КНР:
\begin{itemize}
	\item Liu et al, 1997: Гипотеза о свободном блуждании не отвергается
	\item Wang et al, 2015: Гипотеза о слабой эффективности рынка не отвергается
\end{itemize}
\end{frame}

\begin{frame}
\frametitle{Методология}
\framesubtitle{Спецификация моделей}
\emph{Источник:  D. A. Dickey and W. A. Fuller. Likelihood Ratio Statistics for Autoregressive Time Series with a Unit Root. Econometrica, 49(4):1057–1072, 1981}

\textbf{Модель 1.}
\begin{equation}
	Y_t=\alpha+\rho Y_{t-1}+e_t\quad (t=2,3,\ldots,n),
\end{equation}

\textbf{Модель 2.}
\begin{equation}
	Y_t=\alpha+\beta(t-1-0.5n)+\rho Y_{t-1}+e_t\quad (t=2,3,\ldots,n),
\end{equation}
где $Y_1$ фиксирован и $e_t \sim \mathcal{N}(0,\,\sigma^{2})$.
\end{frame}

\begin{frame}
	\frametitle{Методология}
	\framesubtitle{Отношение правдоподобия и тестовая статистика}
	\textbf{Функция правдоподобия}
	\begin{align}
	\label{eq:likefun}
	log{(L)}=&-0.5(n-1)log(2\pi)-(n-1)log{(\sigma)}-(2\sigma^2)^{-1}\times\\
	\times&\sum_{t=2}^{n}\left[Y_t-\alpha-\beta(t-1-0.5n)-\rho Y_{t-1}\right]^2
	\end{align}
	Если $H_0:(\alpha,\beta,\rho)=(0,0,1)$ не отвергается, то максимум функции правдоподобия достигается в точке
	\begin{equation}
	\hat{\sigma}_0=(n-1)^{(-1)}\sum_{t=2}^n(Y_t-Y_{t-1})^2
	\end{equation}
	Если $H_0$ отвергается в пользу $H_A$, то максимум достигается в точке $(\hat{\sigma}_1^2,\hat{\theta}_\tau^{'})$, где
	\begin{align}
	\hat{\sigma}_1^2&=(n-4)(n-1)^{(-1)}S^2_{e\tau},\\
	\hat{\theta}_\tau&=(\hat{\alpha}_\tau,\hat{\beta}_\tau,\hat{\rho}_\tau)'=(X'X)^{-1}X'Y
	\end{align}
\end{frame}

\begin{frame}
\frametitle{Отношение правдоподобия и тестовые статистики}
Отношение правдоподобия 
\begin{align}
\left[\hat{\sigma}_0^{-1}\hat{\sigma}_1\right]^{n-1}&=\left[1+3(n-4)^{-1}\Phi_2\right],\quad where\\
\Phi_2&=(3S^2_{e\tau})\left[(n-1)\hat{\sigma}_0^2-(n-4)S^2_{e\tau}\right]
\end{align}

\begin{table}
	\centering
	\caption{Тестовые статистики и соответствующие им гипотезы.}
	\begin{tabular}{|l|c|}
		\hline
		\multicolumn{1}{|c|}{$H_0$} & Тестовая статистика \\ \hline
		$(\alpha,\rho)=(0,1)$ & $\Phi_1$ \\ \hline
		$(\alpha,\beta,\rho)=(0,0,1)$ & $\Phi_2$ \\ \hline
		$(\alpha,\beta,\rho)=(\alpha,0,1)$ & $\Phi_3$ \\ \hline
	\end{tabular}
\end{table}
\end{frame}

\begin{frame}
\frametitle{Результаты}
\framesubtitle{Сравнение с исследованием Liu et al, 1997}
\begin{multicols}{2}
\begin{center}
	\textbf{Tumanyants, 2019}
\end{center}
\begin{itemize}
	\item SSE: 1/5/2007-1/2/2019
	\item SZE: 14/9/2012-7/5/2019
	\item $n^{SSE}=2943$
	\item $n^{SZE}=1611$
	\item $\Phi_1^{SSE}=-0.00012$
	\item $\Phi_2^{SSE}=-0.05$
	\item $\Phi_3^{SSE}=0.00021$
	\item $\Phi_1^{SZE}=-0.061$
	\item $\Phi_2^{SZE}=-0.04$
	\item $\Phi_3^{SZE}=0.00155$
	\item \sout{Random walk}
\end{itemize}
\begin{center}
	\textbf{Liu et al, 1997}
\end{center}
\begin{itemize}
	\item SSE: 21/5/1992-16/12/1995
	\item SZE: 21/5/1992-16/12/1995
	\item $n^{SSE}=933$
	\item $n^{SZE}=933$
	\item $\Phi_1^{SSE}=3.126$
	\item $\Phi_2^{SSE}=4.57\,\bigstar$
	\item $\Phi_3^{SSE}=-3.017$
	\item $\Phi_1^{SZE}=3.160$
	\item $\Phi_2^{SZE}=4.374\,\bigstar$
	\item $\Phi_3^{SZE}=-2.95$
	\item Random walk!
\end{itemize}
\end{multicols}
\end{frame}

\begin{frame}
	\frametitle{Результаты}
	\framesubtitle{Сравнение с исследованием Wang et al, 2015}
	\begin{multicols}{2}
		\begin{center}
			\textbf{Tumanyants, 2019}
		\end{center}
	\begin{itemize}
		\item SSE: 1/5/2007-1/2/2019
		\item ADF: -2.09(4) +D -T
		\item PP: -7.96 (10) +D -T
		\item KPSS: 0.099 (12) +D -T*
	\end{itemize}
	\begin{center}
		\textbf{Wang et al, 2015}
	\end{center}
\begin{itemize}
		\item SSE: 1/12/1990-1/3/2013
		\item ADF: -3.420 (0)
		\item PP: -3.446 (0) **
		\item KPSS: 0.158 (12) **
\end{itemize}
\end{multicols}

\begin{multicols}{2}
\begin{itemize}
	\item SZE: 1/5/2007-1/2/2019
	\item ADF: -2.06(5) +D -T
	\item PP: -8.9 (8) +D -T
	\item KPSS: 0.16 (9) +D -T*
\end{itemize}
\begin{itemize}
	\item Итого для SSE и SZE
	\item ADF: $H_0$ не отв. \textbf{EMH!}
	\item PP: $H_0$ отв. \textbf{EMH!}
	\item KPSS: $H_0$ отв. \textbf{EMH!}
\end{itemize}
\end{multicols}
\end{frame}

\begin{frame}[c]
\begin{center}
\frametitle{\LARGE Спасибо за внимание!}

{\LARGE \inserttitle}

\bigskip

{\insertauthor}

\bigskip\bigskip

{\insertinstitute}

\bigskip\bigskip

{\large \insertdate}
\end{center}
\end{frame}

\end{document}
