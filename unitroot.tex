\documentclass[]{article}
\usepackage{lmodern}
\usepackage{amssymb,amsmath}
\usepackage{ifxetex,ifluatex}
\usepackage{fixltx2e} % provides \textsubscript
\ifnum 0\ifxetex 1\fi\ifluatex 1\fi=0 % if pdftex
  \usepackage[T1]{fontenc}
  \usepackage[utf8]{inputenc}
\else % if luatex or xelatex
  \ifxetex
    \usepackage{mathspec}
  \else
    \usepackage{fontspec}
  \fi
  \defaultfontfeatures{Ligatures=TeX,Scale=MatchLowercase}
\fi
% use upquote if available, for straight quotes in verbatim environments
\IfFileExists{upquote.sty}{\usepackage{upquote}}{}
% use microtype if available
\IfFileExists{microtype.sty}{%
\usepackage{microtype}
\UseMicrotypeSet[protrusion]{basicmath} % disable protrusion for tt fonts
}{}
\usepackage[margin=1in]{geometry}
\usepackage{hyperref}
\hypersetup{unicode=true,
            pdftitle={unitroot},
            pdfauthor={Туманянц Артемий},
            pdfborder={0 0 0},
            breaklinks=true}
\urlstyle{same}  % don't use monospace font for urls
\usepackage{color}
\usepackage{fancyvrb}
\newcommand{\VerbBar}{|}
\newcommand{\VERB}{\Verb[commandchars=\\\{\}]}
\DefineVerbatimEnvironment{Highlighting}{Verbatim}{commandchars=\\\{\}}
% Add ',fontsize=\small' for more characters per line
\usepackage{framed}
\definecolor{shadecolor}{RGB}{248,248,248}
\newenvironment{Shaded}{\begin{snugshade}}{\end{snugshade}}
\newcommand{\AlertTok}[1]{\textcolor[rgb]{0.94,0.16,0.16}{#1}}
\newcommand{\AnnotationTok}[1]{\textcolor[rgb]{0.56,0.35,0.01}{\textbf{\textit{#1}}}}
\newcommand{\AttributeTok}[1]{\textcolor[rgb]{0.77,0.63,0.00}{#1}}
\newcommand{\BaseNTok}[1]{\textcolor[rgb]{0.00,0.00,0.81}{#1}}
\newcommand{\BuiltInTok}[1]{#1}
\newcommand{\CharTok}[1]{\textcolor[rgb]{0.31,0.60,0.02}{#1}}
\newcommand{\CommentTok}[1]{\textcolor[rgb]{0.56,0.35,0.01}{\textit{#1}}}
\newcommand{\CommentVarTok}[1]{\textcolor[rgb]{0.56,0.35,0.01}{\textbf{\textit{#1}}}}
\newcommand{\ConstantTok}[1]{\textcolor[rgb]{0.00,0.00,0.00}{#1}}
\newcommand{\ControlFlowTok}[1]{\textcolor[rgb]{0.13,0.29,0.53}{\textbf{#1}}}
\newcommand{\DataTypeTok}[1]{\textcolor[rgb]{0.13,0.29,0.53}{#1}}
\newcommand{\DecValTok}[1]{\textcolor[rgb]{0.00,0.00,0.81}{#1}}
\newcommand{\DocumentationTok}[1]{\textcolor[rgb]{0.56,0.35,0.01}{\textbf{\textit{#1}}}}
\newcommand{\ErrorTok}[1]{\textcolor[rgb]{0.64,0.00,0.00}{\textbf{#1}}}
\newcommand{\ExtensionTok}[1]{#1}
\newcommand{\FloatTok}[1]{\textcolor[rgb]{0.00,0.00,0.81}{#1}}
\newcommand{\FunctionTok}[1]{\textcolor[rgb]{0.00,0.00,0.00}{#1}}
\newcommand{\ImportTok}[1]{#1}
\newcommand{\InformationTok}[1]{\textcolor[rgb]{0.56,0.35,0.01}{\textbf{\textit{#1}}}}
\newcommand{\KeywordTok}[1]{\textcolor[rgb]{0.13,0.29,0.53}{\textbf{#1}}}
\newcommand{\NormalTok}[1]{#1}
\newcommand{\OperatorTok}[1]{\textcolor[rgb]{0.81,0.36,0.00}{\textbf{#1}}}
\newcommand{\OtherTok}[1]{\textcolor[rgb]{0.56,0.35,0.01}{#1}}
\newcommand{\PreprocessorTok}[1]{\textcolor[rgb]{0.56,0.35,0.01}{\textit{#1}}}
\newcommand{\RegionMarkerTok}[1]{#1}
\newcommand{\SpecialCharTok}[1]{\textcolor[rgb]{0.00,0.00,0.00}{#1}}
\newcommand{\SpecialStringTok}[1]{\textcolor[rgb]{0.31,0.60,0.02}{#1}}
\newcommand{\StringTok}[1]{\textcolor[rgb]{0.31,0.60,0.02}{#1}}
\newcommand{\VariableTok}[1]{\textcolor[rgb]{0.00,0.00,0.00}{#1}}
\newcommand{\VerbatimStringTok}[1]{\textcolor[rgb]{0.31,0.60,0.02}{#1}}
\newcommand{\WarningTok}[1]{\textcolor[rgb]{0.56,0.35,0.01}{\textbf{\textit{#1}}}}
\usepackage{graphicx,grffile}
\makeatletter
\def\maxwidth{\ifdim\Gin@nat@width>\linewidth\linewidth\else\Gin@nat@width\fi}
\def\maxheight{\ifdim\Gin@nat@height>\textheight\textheight\else\Gin@nat@height\fi}
\makeatother
% Scale images if necessary, so that they will not overflow the page
% margins by default, and it is still possible to overwrite the defaults
% using explicit options in \includegraphics[width, height, ...]{}
\setkeys{Gin}{width=\maxwidth,height=\maxheight,keepaspectratio}
\IfFileExists{parskip.sty}{%
\usepackage{parskip}
}{% else
\setlength{\parindent}{0pt}
\setlength{\parskip}{6pt plus 2pt minus 1pt}
}
\setlength{\emergencystretch}{3em}  % prevent overfull lines
\providecommand{\tightlist}{%
  \setlength{\itemsep}{0pt}\setlength{\parskip}{0pt}}
\setcounter{secnumdepth}{0}
% Redefines (sub)paragraphs to behave more like sections
\ifx\paragraph\undefined\else
\let\oldparagraph\paragraph
\renewcommand{\paragraph}[1]{\oldparagraph{#1}\mbox{}}
\fi
\ifx\subparagraph\undefined\else
\let\oldsubparagraph\subparagraph
\renewcommand{\subparagraph}[1]{\oldsubparagraph{#1}\mbox{}}
\fi

%%% Use protect on footnotes to avoid problems with footnotes in titles
\let\rmarkdownfootnote\footnote%
\def\footnote{\protect\rmarkdownfootnote}

%%% Change title format to be more compact
\usepackage{titling}

% Create subtitle command for use in maketitle
\newcommand{\subtitle}[1]{
  \posttitle{
    \begin{center}\large#1\end{center}
    }
}

\setlength{\droptitle}{-2em}

  \title{unitroot}
    \pretitle{\vspace{\droptitle}\centering\huge}
  \posttitle{\par}
    \author{Туманянц Артемий}
    \preauthor{\centering\large\emph}
  \postauthor{\par}
      \predate{\centering\large\emph}
  \postdate{\par}
    \date{5/28/2019}


\begin{document}
\maketitle

\begin{Shaded}
\begin{Highlighting}[]
\KeywordTok{adf.test}\NormalTok{(ds)}
\end{Highlighting}
\end{Shaded}

\begin{multicols}{2}\begin{Verbatim}[fontsize=\small]
## Augmented Dickey-Fuller Test
## alternative: stationary
##
## Type 1: no drift no trend
##       lag      ADF p.value
##  [1,]   0 -0.01785   0.639
##  [2,]   1 -0.00484   0.643
##  [3,]   2  0.00119   0.644
##  [4,]   3 -0.00979   0.641
##  [5,]   4 -0.01997   0.638
##  [6,]   5 -0.02217   0.638
##  [7,]   6 -0.01323   0.640
##  [8,]   7 -0.01098   0.641
##  [9,]   8 -0.01031   0.641
## Type 2: with drift no trend
##       lag   ADF p.value
##  [1,]   0 -1.93   0.355
##  [2,]   1 -1.97   0.342
##  [3,]   2 -1.92   0.361
##  [4,]   3 -1.97   0.342
##  [5,]   4 -2.09   0.291
##  [6,]   5 -2.08   0.298
##  [7,]   6 -1.96   0.345
##  [8,]   7 -2.02   0.322
##  [9,]   8 -2.01   0.322
## Type 3: with drift and trend
##       lag   ADF p.value
##  [1,]   0 -1.92   0.611
##  [2,]   1 -1.95   0.597
##  [3,]   2 -1.90   0.618
##  [4,]   3 -1.95   0.597
##  [5,]   4 -2.08   0.542
##  [6,]   5 -2.06   0.549
##  [7,]   6 -1.94   0.600
##  [8,]   7 -2.00   0.575
##  [9,]   8 -2.00   0.576
## ----
## Note: in fact, p.value = 0.01 means p.value <= 0.01
\end{Verbatim}\end{multicols}

\begin{Shaded}
\begin{Highlighting}[]
\KeywordTok{adf.test}\NormalTok{(ds2)}
\end{Highlighting}
\end{Shaded}

\begin{multicols}{2}\begin{Verbatim}[fontsize=\small]
## Augmented Dickey-Fuller Test
## alternative: stationary
##
## Type 1: no drift no trend
##      lag     ADF p.value
## [1,]   0 -0.0718   0.623
## [2,]   1 -0.1129   0.611
## [3,]   2 -0.1556   0.599
## [4,]   3 -0.1839   0.591
## [5,]   4 -0.1937   0.588
## [6,]   5 -0.2380   0.575
## [7,]   6 -0.2356   0.576
## [8,]   7 -0.2375   0.576
## Type 2: with drift no trend
##      lag   ADF p.value
## [1,]   0 -2.04   0.313
## [2,]   1 -1.99   0.330
## [3,]   2 -1.93   0.356
## [4,]   3 -2.01   0.326
## [5,]   4 -2.04   0.310
## [6,]   5 -2.06   0.306
## [7,]   6 -2.01   0.323
## [8,]   7 -2.20   0.246
## Type 3: with drift and trend
##      lag   ADF p.value
## [1,]   0 -2.20   0.494
## [2,]   1 -2.13   0.519
## [3,]   2 -2.05   0.554
## [4,]   3 -2.11   0.528
## [5,]   4 -2.15   0.514
## [6,]   5 -2.14   0.517
## [7,]   6 -2.10   0.534
## [8,]   7 -2.28   0.457
## ----
## Note: in fact, p.value = 0.01 means p.value <= 0.01
\end{Verbatim}\end{multicols}

\begin{Shaded}
\begin{Highlighting}[]
\KeywordTok{pp.test}\NormalTok{(ds)}
\end{Highlighting}
\end{Shaded}

\begin{multicols}{2}\begin{Verbatim}[fontsize=\small]
## Phillips-Perron Unit Root Test
## alternative: stationary
##
## Type 1: no drift no trend
##  lag    Z_rho p.value
##    9 -0.00246   0.692
## -----
##  Type 2: with drift no trend
##  lag Z_rho p.value
##    9 -7.96   0.293
## -----
##  Type 3: with drift and trend
##  lag Z_rho p.value
##    9 -7.93   0.588
## ---------------
## Note: p-value = 0.01 means p.value <= 0.01
\end{Verbatim}\end{multicols}

\begin{Shaded}
\begin{Highlighting}[]
\KeywordTok{pp.test}\NormalTok{(ds2)}
\end{Highlighting}
\end{Shaded}

\begin{multicols}{2}\begin{Verbatim}[fontsize=\small]
## Phillips-Perron Unit Root Test
## alternative: stationary
##
## Type 1: no drift no trend
##  lag    Z_rho p.value
##    8 -0.00591   0.691
## -----
##  Type 2: with drift no trend
##  lag Z_rho p.value
##    8  -8.9   0.239
## -----
##  Type 3: with drift and trend
##  lag Z_rho p.value
##    8 -9.85   0.468
## ---------------
## Note: p-value = 0.01 means p.value <= 0.01
\end{Verbatim}\end{multicols}

\begin{Shaded}
\begin{Highlighting}[]
\KeywordTok{kpss.test}\NormalTok{(ds)}
\end{Highlighting}
\end{Shaded}

\begin{multicols}{2}\begin{Verbatim}[fontsize=\small]
## KPSS Unit Root Test
## alternative: nonstationary
##
## Type 1: no drift no trend
##  lag   stat p.value
##   12 0.0723     0.1
## -----
##  Type 2: with drift no trend
##  lag   stat p.value
##   12 0.0974     0.1
## -----
##  Type 1: with drift and trend
##  lag  stat p.value
##   12 0.099     0.1
## -----------
## Note: p.value = 0.01 means p.value <= 0.01
##     : p.value = 0.10 means p.value >= 0.10
\end{Verbatim}
\end{multicols}

\begin{Shaded}
\begin{Highlighting}[]
\KeywordTok{kpss.test}\NormalTok{(ds2)}
\end{Highlighting}
\end{Shaded}

\begin{multicols}{2}\begin{Verbatim}[fontsize=\small]
## KPSS Unit Root Test
## alternative: nonstationary
##
## Type 1: no drift no trend
##  lag   stat p.value
##    9 0.0701     0.1
## -----
##  Type 2: with drift no trend
##  lag stat p.value
##    9 0.16     0.1
## -----
##  Type 1: with drift and trend
##  lag   stat p.value
##    9 0.0728     0.1
## -----------
## Note: p.value = 0.01 means p.value <= 0.01
##     : p.value = 0.10 means p.value >= 0.10
\end{Verbatim}\end{multicols}


\end{document}
